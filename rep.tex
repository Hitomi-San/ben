\documentclass[jafontsize=9pt,twocolumn]{jlreq}

\usepackage{luatexja}
\ltjsetparameter{jacharrange={-2,-3,-8}}
\usepackage[no-math,match,deluxe]{luatexja-preset}

\usepackage[T1]{fontenc}
\usepackage{textcomp}
\usepackage{luatexja-otf}

\usepackage{pxrubrica}
\usepackage{indentfirst}
\usepackage{tcolorbox}

\usepackage{tikz}
\usetikzlibrary{plotmarks}

\usepackage[colorlinks,unicode=true]{hyperref}
\hypersetup{linkcolor=blue,urlcolor=teal}
\setcounter{tocdepth}{2}

\usepackage{yhmath,amsmath,mathtools,amssymb,mathrsfs,rsfso,mleftright}
\usepackage[math]{kurier}
\usepackage[euler-digits]{eulervm}
\usepackage[scaled]{beramono}
\DeclareMathAlphabet{\mathtt}{T1}{fvm}{m}{n}
\DeclareMathAlphabet{\mathsf}{T1}{uop}{m}{n}
\allowdisplaybreaks[4]

\setmainjfont{FOT-RodinNTLGPro-DB}[
	AltFont={
			{
				Range={"0370-"04FF,"3000-"30FF,"FF00-"FFEF},
				Font=FOT-MatissePro-DB,
			}
		},
	BoldFeatures={
		Font=FOT-RodinNTLGPro-EB,
		AltFont={
				{
					Range={"0370-"04FF,"3000-"30FF,"FF00-"FFEF},
					Font=FOT-MatissePro-EB,
				}
			},
	},
	YokoFeatures={JFM=jlreq},   % jlreqのJFMを維持する
	TateFeatures={JFM=jlreqv},  % https://qiita.com/zr_tex8r/items/91ae1dcc9c3afce7fa8c
]
\setsansjfont{FOT-RodinNTLGPro-DB}[
	AltFont={
			{
				Range={"0370-"04FF,"3000-"30FF,"FF00-"FFEF},
				Font=FOT-MatissePro-DB,
			},
		},
	BoldFeatures={
		Font=FOT-RodinNTLGPro-EB,
		AltFont={
				{
					Range={"0370-"04FF,"3000-"30FF,"FF00-"FFEF},
					Font=FOT-MatissePro-EB,
				}
			},
	},
	YokoFeatures={JFM=jlreq},   % jlreqのJFMを維持する
	TateFeatures={JFM=jlreqv},  % https://qiita.com/zr_tex8r/items/91ae1dcc9c3afce7fa8c
]
\setmonofont[
	Ligatures=TeXReset,
]{Noto Sans Mono CJK JP}
\setmonojfont[
	Ligatures=TeXReset,
]{Noto Sans Mono CJK JP}
\setmainfont{Palatino}
\setsansfont{Optima}

\usepackage{listings}
\lstset{
	basicstyle={\footnotesize\ttfamily},
	commentstyle={\color[gray]{.5}}
}


\begin{document}
\title{とある授業のレポート}
\author{ひとみさん}

\maketitle

\begin{abstract}
Benfordの法則とは、自然界に現れる数字を集めたときに、一番上の位の数字の出現頻度は対数的な分布をし、
1が最も出現頻度が高いという法則である。この法則に、気象計測で得た膨大な数値のデータが従っているかを
確認した。
\end{abstract}

\section{はじめに}

Benfordの法則とは、次のような法則である。

\begin{tcolorbox}[title={Benfordの法則}]
自然界に出現する数値で、最上位の数字$d$が$n$と等しくなる確率$\Pr[d=n]$は
\[\log_{10}\mleft[\frac{n+1}{n}\mright]\]
\end{tcolorbox}

この法則\footnote{実際にはこの法則を、最上位の数字だけではなく、上数桁の出現頻度へと
拡張したものが使われている。}は会計の現場で、不正を検証するために実際に利用されている。
しかしながら、Benfordの法則の存在を知っていても、それを実際に確かめることができるほど
多くの数値のデータを入手する機会はなかなかない。そこで、8号館の気象データをもちいて
Benfordの法則を検証することとした。


\section{Benfordの法則とは}

Benfordの法則とは、自然界に出てくる多くの数値の最初の桁の分布が、一様分布ではなく
対数的な分布になっているという法則である。ここで、対数的な分布(logarithmic distribution)とは、
対数目盛の各区間の広さ$\log_{10}[d+1]-\log_{10}[d]=\log_{10}[1+1/d]$に比例した
分布という意味である。

\subsection{Benfordの法則の実例}

8号館のデータについて検証する前に、Benfordの法則が成り立つ実例と、その簡易な証明について記す。

\subsubsection{数列$\{2^k\}$}

数列$\{2^k\}$に現れる数値はBenfordの法則に従う。$1\leq k\leq 100$の範囲で、最上位の桁の
数字の現れる頻度を表\ref{pow-tbl}と図\ref{pow-graph}に示した。図を見ると、100個という
少ないサンプル数でも、非常によくBenfordの法則に従っているのがわかる。

数列$\{2^k\}$に現れる数値はBenfordの法則に従うことついての簡易な証明を以下に示す。

$2^k$の最上位の数字を$d$とし、$m$を整数とすると、
\begin{gather*}
d\times10^m\leq2^k<(d+1)\times10^m
\intertext{常用対数を取ると、}
m+\log[d]\leq k\log[2]<m+\log[d+1] \\
\frac{m+\log[d]}{\log[2]}\leq k<\frac{m+\log[d+1]}{\log[2]}
\end{gather*}
$d=d_n$のとき、不等式を満たす$k$の幅$x_n$を求める。
\begin{align*}
x_n&=\frac{m+\log[d_n]}{\log[2]}-\frac{m+\log[d_n+1]}{\log[2]} \\
&=\frac{1}{\log[2]}\cdot\log\mleft[\frac{d_n+1}{d_n}\mright]
\end{align*}
これより、$x_n$は$m$(桁数)によらない事がわかる。

次に、$\sum^9_{i=1}[x_i]$を求めておく。
\begin{align*}
\sum^9_{i=1}[x_i]&=\frac{1}{\log[2]}\sum^9_{i=1}\log\mleft[\frac{d_i+1}{d_i}\mright] \\
&=\frac{1}{\log[2]}\cdot\log\mleft[\frac{2}{1}\cdot\frac{3}{2}\cdot\cdots\cdot\frac{10}{9}\mright] \\
&=\frac{1}{\log[2]}\cdot\log[10]=\frac{1}{\log[2]}
\end{align*}

$k$は一様分布なので、$d=n$となる確率$\Pr[d=n]$は、
\begin{align*}
\Pr[d=n]&=\frac{x_n}{\sum^9_{i=1}[x_i]} \\
&=\mleft.\frac{1}{\log[2]}\cdot\log\mleft[\frac{n+1}{n}\mright]\middle/\frac{1}{\log[2]}\mright. \\
&=\log\mleft[\frac{d+1}{d}\mright]
\end{align*}

\subsubsection{フィボナッチ数列$\{F_k\}$}

フィボナッチ数列$\{F_k\}$は以下で定義される数列である。
\begin{align*}
\begin{cases}
	F_1&=1\\
	F_2&=1\\
	F_{n+2}&=F_{n+1}+F_n
\end{cases}
\end{align*}

フィボナッチ数列もBenfordの法則に従う。フィボナッチ数列についても、$1\leq k\leq 100$の範囲で、最上位の桁の
数字の現れる頻度を表\ref{pow-tbl}と図\ref{pow-graph}に示した。こちらもよくBenfordの法則に従っている。
このことについての証明も記す。

フィボナッチ数列の漸化式から一般項を求めると、
\[F_n=\frac{1}{\sqrt{5}}\mleft(\mleft(\frac{1+\sqrt{5}}{2}\mright)^n-\mleft(\frac{1-\sqrt{5}}{2}\mright)^n\mright)\]
である。

この式の第2項$\mleft(\frac{1-\sqrt{5}}{2}\mright)^n/\sqrt{5}$は、$n=1$のときに最大値
$1/\sqrt{5}\approx 0.447$を取るので、第2項を省略した
\[F_n\approx \frac{\phi^n}{\sqrt{5}}, \qquad \phi=\frac{1+\sqrt{5}}{2}\,\text{(黄金比)}\]
は$F_n$の値を0.447以下の誤差で、特に$n>4$のときは$1\%$以下の誤差で与える近似式である。
今は最上位の桁の数字しか着目していないので、近似的に$F_n=\phi^n/\sqrt{5}$とする。

このように近似すると、$\{2^k\}$と同じ議論をすることによって、同じ確率を得ることができる。
\[\Pr[d=n]=\log_{10}\mleft[\frac{n+1}{n}\mright]\]


\begin{table}[b]
\caption{$1\leq n\leq100$の範囲における最上位桁の出現頻度}
\label{pow-tbl}
\centering
\begin{tabular}{c|rr}
\hline
数字&$2^n$&$F_n$\\
\hline\hline
$1$&$30$&$30$\\
$2$&$17$&$18$\\
$3$&$13$&$13$\\
$4$&$10$&$9$\\
$5$&$7$&$8$\\
$6$&$7$&$6$\\
$7$&$6$&$5$\\
$8$&$5$&$7$\\
$9$&$5$&$4$\\
\hline
\end{tabular}
\end{table}

\begin{figure}[b]
\centering
\begin{tikzpicture}[x=.08\linewidth,y=.05\linewidth,yscale=20,domain=0:10]
\draw[->](-1,0)--(10,0)node[right]{$x$};\draw(5,-0.05)node[below]{最上位に出現する数字};
\foreach\x in {1,...,9} \draw(\x,0.01)--(\x,-0.01)node[below]{\small$\x$};
\draw[->](0,-0.05)--(0,0.55)node[above]{$y$};\draw(-1,0.25)node[left]{\hbox{\tate 割合}};
\foreach\x in {0.05,0.1,0.15,0.2,0.25,0.3,0.35,0.4,0.45,0.5} \draw(0.2,\x)--(-0.2,\x);
\foreach\x in {0.1,0.2,0.3,0.4} \draw(-0.2,\x)node[left]{\small$\x$};
\draw(0,0) node[below left]{$\mathrm{O}$};
\draw(0.6,0.4) node[above right]{$\displaystyle y=\log_{10}\mleft[1+\frac{1}{x}\mright]$};
\draw [domain=0.4:10,samples=100,smooth] plot(\x,{ln((\x+1)/\x)/ln(10)});
\draw[color=red,thick,mark=text,text mark={\scalebox{1}[0.05]{●}}] plot file{2pown.tbl};
\draw[color=red](3,0.3)node[left]{$2^n$};
\draw[color=blue,thick,mark=text,text mark={\scalebox{1}[0.05]{◆}}] plot file{fn.tbl};
\draw[color=blue](3,0.3)node[right]{$F_n$};
\end{tikzpicture}
\caption{$1\leq n\leq100$の範囲における最上位桁の出現頻度}
\label{pow-graph}
\end{figure}


\section{8号館のデータを解析する}

8号館の気象データをもちいてBenfordの法則を検証することとした。検証に用いたデータは、4月19日から
7月8日までのデータである。検証の対象は日付と時刻を除く全てのデータである。データを解析するのに用いた
\TeX コード、Pythonコード、Rubyコードを\ref{tex}、\ref{python}、\ref{ruby}に示した。
このコードの開発には、工学院や工学部の方に協力いただいた。
また、このコードで解析した結果を表\ref{8-tbl}と図\ref{8-graph}に示した。
\TeX での実装と、pythonとRubyでの実装とで結果が異なるのは、\TeX で処理するためにはデータの前処理
が必要で、その前処理の方法が良くなかったからなのではないかと考えている。

\begin{table}[b]
\caption{8号館のデータに現れる最上位桁の出現頻度}
\label{8-tbl}
\centering
\begin{tabular}{c||rr|rrr}
\hline
数字&\TeX&割合$[\%]$&Python&Ruby&割合$[\%]$\\
\hline\hline
$1$&$133358$&$49.9$&$139375$&$139375$&$50.028$\\
$2$&$52413$&$18.8$&$46462$&$46462$&$16.677$\\
$3$&$8957$&$3.2$&$8008$&$8008$&$3.162$\\
$4$&$12325$&$4.4$&$13558$&$13558$&$4.867$\\
$5$&$14052$&$5.0$&$15496$&$15496$&$5.562$\\
$6$&$13366$&$4.8$&$12263$&$12263$&$4.402$\\
$7$&$12939$&$4.6$&$13197$&$13197$&$4.737$\\
$8$&$15063$&$5.4$&$14403$&$14403$&$5.170$\\
$9$&$15814$&$5.6$&$15034$&$15034$&$5.396$\\
\hline
計&$278278$&$100.0$&$278596$&$278596$&$100.000$\\
\hline
\end{tabular}
\end{table}

\begin{figure}[b]
\label{8-graph}
\centering
\begin{tikzpicture}[x=.08\linewidth,y=.05\linewidth,yscale=20,domain=0:10]
\draw[->](-1,0)--(10,0)node[right]{$x$};\draw(5,-0.05)node[below]{最上位に出現する数字};
\foreach\x in {1,...,9} \draw(\x,0.01)--(\x,-0.01)node[below]{\small$\x$};
\draw[->](0,-0.05)--(0,0.55)node[above]{$y$};\draw(-1,0.25)node[left]{\hbox{\tate 割合}};
\foreach\x in {0.05,0.1,0.15,0.2,0.25,0.3,0.35,0.4,0.45,0.5} \draw(0.2,\x)--(-0.2,\x);
\foreach\x in {0.1,0.2,0.3,0.4} \draw(-0.2,\x)node[left]{\small$\x$};
\draw(0,0) node[below left]{$\mathrm{O}$};
\draw(0.5,0.5) node[above right]{$\displaystyle y=\log_{10}\mleft[\frac{x+1}{x}\mright]$};
\draw[domain=0.4:10,samples=100,smooth] plot (\x,{ln((\x+1)/\x)/ln(10)});
\draw[color=red,thick,mark=text,text mark={\tiny\scalebox{1}[0.05]{●}}] plot file{tex.tbl};
\draw[color=red](2,0.3)node[above right]{\TeX};
\draw[color=blue,thick,mark=text,text mark={\tiny\scalebox{1}[0.05]{◆}}] plot file{py.tbl};
\draw[color=blue](2,0.3)node[below right]{Python, Ruby};
\end{tikzpicture}
\caption{8号館のデータに現れる最上位桁の出現頻度}
\end{figure}

グラフを見るだけでは、最上位桁の頻度分布が対数的になっているかわからない。特に、グラフと比較すると
1の出現する頻度が多すぎるようにも感じられる。
そこでPythonのライブラリを用いて、頻度分布が対数的になっているか$\chi^2$検定を実施した。
その結果は次のようになった。

\begin{align*}
\chi^2&=68084.00878444461\\ p&=0.0
\end{align*}

$p$値は実際の分布が、期待される分布にどれだけ従っているかを表す指標である。この場合、$p$値は$0$であるから
帰無仮説は棄却される。したがって、8号館のデータはBenfordの法則に従っていないと言える。

\subsection{気温データに限定して検証する}

8号館のデータを解析しても、Benfordの法則に従っていなかった理由を考えたところ、次の点が原因なのではないか
と考えた。それは、解析の対象となるデータの中に気圧のデータが含まれており、気圧の値の最上位桁は
殆どの場合$1$、まれに$9$になるのみで、明らかにBenfordの法則に従わないため、それの影響をうけて
いるのではないかということだ。

そこで、次に気温を表すデータのみに絞って、Pythonで解析を行った。
その結果を表\ref{8t-tbl}と図\ref{8t-graph}に示す。また、先と同じように$\chi^2$検定も行った。
その結果は次のようである。

\begin{align*}
\chi^2&=24282.84808321144\\ p&=0.0
\end{align*}

こちらも$p$値が$0$なので、気温データのみで解析しても、Benfordの法則に従っているとは言えなかった。


\begin{table}[b]
\caption{8号館の気温データに現れる最上位桁の出現頻度}
\label{8t-tbl}
\centering
\begin{tabular}{c||rr}
\hline
数字&度数&割合$[\%]$\\
\hline\hline
$1$&$22122$&$63.22$\\
$2$&$5817$&$16.62$\\
$3$&$94$&$0.27$\\
$4$&$99$&$0.28$\\
$5$&$301$&$0.86$\\
$6$&$550$&$1.57$\\
$7$&$1742$&$4.98$\\
$8$&$2266$&$6.48$\\
$9$&$2271$&$6.49$\\
\hline
計&$34992$&$100.00$\\
\hline
\end{tabular}
\end{table}

\begin{figure}[b]
\label{8t-graph}
\centering
\begin{tikzpicture}[x=.08\linewidth,y=.05\linewidth,yscale=20,domain=0:10]
\draw[->](-1,0)--(10,0)node[right]{$x$};\draw(5,-0.05)node[below]{最上位に出現する数字};
\foreach\x in {1,...,9} \draw(\x,0.01)--(\x,-0.01)node[below]{\small$\x$};
\draw[->](0,-0.05)--(0,0.55)node[above]{$y$};\draw(-1,0.25)node[left]{\hbox{\tate 割合}};
\foreach\x in {0.05,0.1,0.15,0.2,0.25,0.3,0.35,0.4,0.45,0.5} \draw(0.2,\x)--(-0.2,\x);
\foreach\x in {0.1,0.2,0.3,0.4} \draw(-0.2,\x)node[left]{\small$\x$};
\draw(0,0) node[below left]{$\mathrm{O}$};
\draw(0.5,0.5) node[above right]{$\displaystyle y=\log_{10}\mleft[\frac{x+1}{x}\mright]$};
\draw[domain=0.4:10,samples=100,smooth] plot (\x,{ln((\x+1)/\x)/ln(10)});
\draw[color=blue,thick,mark=text,text mark={\tiny\scalebox{1}[0.05]{◆}}] plot file{temp.tbl};
\end{tikzpicture}
\caption{8号館の気温データに現れる最上位桁の出現頻度}
\end{figure}


\section{まとめ}

8号館から得た気象データは、Benfoedの法則に従っているとは言えないことがわかった。
逆に言えば、8号館から得た気象データは、Benfordの法則に従うような確率分布をもっていないということである。

Benfordの法則に従うような確率分布は、スケール不変な確率分布であるということが知られている。
スケール不変とは、観測対象$F$について、任意のスケール変換$x \to \lambda x$に対し、
\[F(\lambda x) = \mu F(x)\]
を満たす定数$\mu$が存在するような$F$のことである。スケール不変なものの例として、べき乗則
$F(x)=cx^\alpha$があげられる。
8号館の気象データがBenfordの法則に従っていないということは、8号館の気象データはスケール不変ではない
ということでもある。

今回は4月から7月までのデータでしか解析を行わなかったが、通年のデータで解析を行ったらどのような結果に
なるのかの検証も待たれる。

\section{参考文献}
\begin{itemize}
\item \url{http://zakii.la.coocan.jp/enumeration/64_benford.htm}
\item \url{https://ja.wikipedia.org/w/index.php?title=%E3%83%99%E3%83%B3%E3%83%95%E3%82%A9%E3%83%BC%E3%83%89%E3%81%AE%E6%B3%95%E5%89%87&oldid=68691651}
\item \url{https://ja.wikipedia.org/w/index.php?title=%E3%82%B9%E3%82%B1%E3%83%BC%E3%83%AB%E4%B8%8D%E5%A4%89%E6%80%A7&oldid=63430785}
\end{itemize}

\appendix
\onecolumn
\section{8号館のデータを解析するのに用いた\TeX コード\label{tex}}
この\TeX ソースをコンパイルするために、気象データのファイルを表計算ソフトなどで処理し、以下の要件を
満たすようにファイルを編集する必要がある。なお、そのファイル名はdata2.texとする。
\begin{itemize}
\item 解析の対象としない、日付や時刻をあらわす列を削除する。
\item 全ての数値を指数表記にする。
\item コンマ区切り(CSV形式)にする。
\item 全ての行頭に\textbackslash counttop:と書く。
\end{itemize}

\lstinputlisting[caption={test.tex},language={TeX},backgroundcolor={\color[gray]{.9}}]{test.tex}
\lstinputlisting[caption={counttop.sty},language={TeX}]{counttop.sty}
\pagebreak

\section{8号館のデータを解析するのに用いたpythonコード\label{python}}
\lstinputlisting[language={python}]{op3.py}
\pagebreak

\section{8号館のデータを解析するのに用いたRubyコード\label{ruby}}
\lstinputlisting[language={ruby}]{benford.rb}


\end{document}